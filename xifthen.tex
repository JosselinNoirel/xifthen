\documentclass{article}

\usepackage{etex}
\usepackage{xifthen}
\usepackage[utf8]{inputenc}
\usepackage[T1]{fontenc}
\usepackage[warn]{textcomp}
\usepackage[a4paper, pdftex, margin=1.5in]{geometry}
\usepackage{lmodern}
\usepackage{microtype}
\usepackage{xcolor}
\usepackage{url}
\usepackage[bottom,hang]{footmisc}
\usepackage[babel]{csquotes}
\usepackage[british]{babel}
\usepackage{hyperref}

\newcommand*{\cmd}[1]{\tex{\textbackslash #1}}
\newcommand*{\tex}{\texttt}
\newenvironment*{texcode}{\list{}{}\item\ttfamily}{\endlist}
\newcommand*{\marg}[1]{\{\meta{#1}\}}
\newcommand*{\oarg}[1]{[\meta{#1}]}
\newcommand*{\meta}[1]{\textnormal{\textlangle#1\textrangle}}

\setcounter{secnumdepth}{-\maxdimen}

\makeatletter
\newcommand*{\1}{$_1\m@th$}
\newcommand*{\2}{$_2\m@th$}

% Borrowed from doc.sty
\newcommand*{\getfileinfo}[1]{%
  \def \filename {#1}%
  \def \@tempb ##1 ##2 ##3\relax ##4\relax {%
    \def \filedate    {##1}%
    \def \fileversion {##2}%
    \def \fileinfo    {##3}%
  }%
  \edef \@tempa {\csname ver@#1\endcsname}%
  \expandafter \@tempb \@tempa \relax ? ? \relax \relax
}
\makeatother

\newenvironment*{texdescription}{%
  \list{}{%
    \setlength{\labelwidth}{0pt}%
    \setlength{\itemindent}{-\leftmargin}%
    \setlength{\itemsep}{0pt}%
    \renewcommand*{\makelabel}[1]%
      {\hspace{\labelsep}\normalfont\tex{\texmonospaces ##1}}%
  }%
}{%
  \endlist
}

\newcommand*{\pack}{\textsf}
\newcommand*{\true}{\emph{true}}
\newcommand*{\false}{\emph{false}}

\newtest \condition [2]{%
  \cnttest{(#1)*(#2)}>{100}%
  \AND
  \cnttest{((#1)+(#2))*2}<{60}%
}

\getfileinfo{xifthen.sty}
\title  {The \pack{xifthen} package}
\author {Josselin Noirel\footnote{This document corresponds to version
  \fileversion\ (\filedate) of \pack{\filename}.}\\%
\url{https://github.com/JosselinNoirel/xifthen}}

\begin{document}

\maketitle

\begin{abstract}
  This package implements new commands that can be used within the
  first argument of \pack{ifthen}'s \cmd{ifthenelse} to test whether a
  string is void or not, if a command is defined or equivalent to
  another.  It includes also the possibility to make use of the
  complex expressions introduced by the package~\pack{calc}, together
  with the ability of defining new commands to handle complex tests.
  This package requires the \eTeX{} features.
\end{abstract}

\tableofcontents

\section{What's new}

\begin{itemize}
\item [1.1] Now \cmd{cnttest} and \cmd{dimtest} accept \tex{<=} and \tex{>=}.
\item I~renamed \cmd{terminateswith} in \cmd{endswith}.

\addvspace{\baselineskip}

\item [1.2] Corrected a~bug related to a~bad interaction between
  new~tests and ifthen's replacement macro (credits go to MPG \&
  P.~Albar\`ede).

\addvspace{\baselineskip}

\item [1.3] Removed a spurious space (thanks to Ulrike Fisher).

\item [1.4.0] Removed the reliance on the \cmd{etex} package, following
      an exmail exchange with David Carlisle and Maïeul Rouquette.
      (I also updated the documentation and pushed the files on GitHub.)
\end{itemize}

\section{\pack{ifthen}'s interface}

\subsection{Declaring and setting booleans}

You can declare boolean (presumably in the preamble of your document) with
%
\begin{texcode}
\cmd{newboolean}\marg{boolean}
\end{texcode}
%
where \meta{boolean} is a name made up of alphanumeric characters.
For instance, 
%
\begin{texcode}
\cmd{newboolean}\{appendix\}\\
\cmd{first}\{appendix\}
\end{texcode}
%
Then your boolean is ready to be set with
%
\begin{texcode}
\cmd{setboolean}\marg{boolean}\marg{truth value}
\end{texcode}
%
where \meta{truth value} can be \tex{true} or \tex{false}.

\subsection{Executing conditional code}

The general syntax is inherited of that of the package~\pack{ifthen}:
%
\begin{texcode}
  \cmd{ifthenelse}\marg{test expression}\marg{true code}\marg{false code}
\end{texcode}
%
Evaluates the \meta{test expression} and executes \meta{true code} if the
test turns out to be true and \meta{false code} otherwise.  \pack{ifthen}
provides the tests explained in the next paragraphs.

\paragraph{Value of a~boolean}

You can use the value of a~boolean you declared, or the value of
a~primitive boolean of \TeX\footnote{The primitive booleans include:
  \tex{mmode} (Are we in math mode?), \tex{hmode} (Are we in
  horizontal mode?), \tex{vmode} (Are we in vertical mode?),~etc.}
%
\begin{texcode}
  \cmd{boolean}\marg{boolean}
\end{texcode}

\paragraph{Tests on integers}

To test whether an~integer is equal to, strictly less than, or
strictly greater than, you write the expression straightforwardly.
%
\begin{list}{}{}
\item \tex{\meta{value\1} = \meta{value\2}}
\item \tex{\meta{value\1} < \meta{value\2}}
\item \tex{\meta{value\1} > \meta{value\2}}
\item \tex{\textbackslash isodd\marg{number}}
\end{list}

\paragraph{Tests on lengths}

There exist similar tests for the lengths, but you need in this case
to surround the whole expression with \cmd{lengthtest}.
%
\begin{list}{}{}
\item \tex{\cmd{lengthtest}\{\meta{dimen\1} = \meta{dimen\2}\}}
\item \tex{\cmd{lengthtest}\{\meta{dimen\1} < \meta{dimen\2}\}}
\item \tex{\cmd{lengthtest}\{\meta{dimen\1} > \meta{dimen\2}\}}
\end{list}

\paragraph{Tests on commands}
 
You can test if a command is undefined\footnote{This test differs from
  \cmd{@ifundefined} in that it takes a~real command---and not
  a~command name---as argument, and also in that command which is let
  equal to \cmd{relax} is not considered undefined.}.
%
\begin{list}{}{}
\item \tex{\cmd{isundefined}\marg{command}}
\end{list}

\paragraph{Tests on character strings}

You want to know whether two character strings are equal?  Use:
%
\begin{texcode}
\cmd{equal}\marg{string\1}\marg{string\2}
\end{texcode}
%
Remark that the two arguments are fully expanded.  In other words, it
is the result of the expansion of the macros that is compared.  This
behaviour also entails a~moving argument and you should protect
fragile command to avoid bizarre errors\footnote{Practically, the
  fact that the content is expanded, means that if the macro \cmd{bar}
  is defined as \tex{\cmd{baz}\{o\}}, and the command \cmd{baz} is defined
  as \tex{f\#1\#1}, then \tex{\cmd{equal}\{\cmd{bar}\}\{foo\}} turns out to
  be true, because \cmd{bar} eventually expands into \tex{foo}.  This is
  usually the desired behaviour.}.

\paragraph{Building more elaborated expressions}

You can build more sophisticated expressing using the \cmd{AND}
(conjunction), \cmd{OR} (disjunction), and \cmd{NOT} (negation)
operators\footnote{Lowercase versions of these commands also exist but
  we advise the user to stick to the uppercase ones because \cmd{or}
  is part of \TeX's syntax.}.
%
\begin{list}{}{}
\item \meta{expression\1} \cmd{AND}  \meta{expression\2}
\item \meta{expression\1} \cmd{OR} \ \meta{expression\2}
\item \cmd{NOT} \meta{expression}
\end{list}
%
The evaluation is lazy, meaning that if you write
%
\begin{texcode}
\meta{expression\1} \cmd{AND} \meta{expression\2}
\end{texcode}
%
then \meta{expression\2} won't be evaluated if \meta{expression\1} is
true\footnote{The devil is in the details, however: \pack{ifthen} works by
  reading its argument twice.  The tests are evaluated on the second
  pass, but the expansion is performed on the first one, regardless of
  the truth value.}.

There is not precedence rules: the argument is read from left to right
and \cmd{NOT} applies to the very next test.  When the precedence must
be changed you can use the parentheses:
%
\begin{texcode}
\cmd{(}\meta{expression}\cmd{)}
\end{texcode}

\section{New tests}

After this brief review of \pack{ifthen}'s principles, we introduce
the new tests provided by \pack{xifthen}.

%%%%%%%%%%%%%%%%%%%%%%%%%%%%%%%%%%%%%%%%%%%%%%%%%%%%%%%%%%%%%%%%%
\paragraph{Tests on integers}

One of the drawback of \TeX's tests and of \cmd{ifthen}'s as well, is
the impossibility to use \pack{calc}'s syntax in it.  The
\cmd{numexpr} primitive of \eTeX\ somehow allows to overcome this
difficulty but it is not well documented and normal users are
certainly more familiar with the capabilities offered by \pack{calc}.
The \pack{xifthen} package allows to use \pack{calc}-valid expressions
via the new test \cmd{cnttest}.  The syntax is as follows:
%
\begin{texcode}
  \cmd{cnttest}\marg{counter expression\1}\marg{comparison}%
               \marg{counter expression\2}
\end{texcode}
%
It evaluates the two counter expressions, compares them, and returns
the value of the test.  The comparison can be one of the following
sequences \tex{<}, \tex{>}, \tex{=}, \tex{<=},~or~\tex{>=}.

\paragraph{Tests on lengths}

The similar test has been designed for the lengths and dimensions:
%
\begin{texcode}
  \cmd{dimtest}\marg{dimen expression\1}\marg{comparison}%
               \marg{dimen expression\2}
\end{texcode}
%
It evaluates the two dimension expressions, compares them, and returns
the value of the test.  The comparison can be one of the following
sequences \tex{<}, \tex{>}, \tex{=}, \tex{<=},~or~\tex{>=}.

\paragraph{Tests on commands}

We define a~companion of \cmd{isundefined} that uses a~command name
rather than a~command\footnote{If you are stuck with the distinction
  between `command' and `command name', let me explain it further with
  an example: the command name of the command \cmd{foo} is \tex{foo}.
  This is sometimes more convenient to use the command name than the
  name.  Still, this functionality is probably intended more for
  experienced programmers who want to use the niceties of
  \pack{ifthen} and \pack{xifthen}.}.
%
\begin{texcode}
  \cmd[1,syntax]{isnamedefined}{command name}
\end{texcode}
%
Returns \true\ if the command \cmd{\meta{command name}} is defined\footnote{%
  Uses \tex{\cmd{ifcsname}\dots\allowbreak\cmd{endcsname}} internally
  and not \cmd{@ifundefined}.}.

Sometimes, you need to compare two macros \cmd{foo} and \cmd{bar} and
test whether they are actually the same macro.
%
\begin{texcode}
  \cmd{isequivalentto}\marg{command\1}\marg{command\2}
\end{texcode}
%
Corresponds to the \cmd{ifx} test: it returns \true\ when the two
commands are exactly equivalent (same definition, same number of
arguments, same prefixes, etc., otherwise \false\ is returned).

\paragraph{Tests on character strings}

Very often, we see people using \cmd[2]{equal}{\#1}{} in their command
definitions (for instance, to test whether an optional argument had
been passed to their macro).  A~more efficient test can be used:
%
\begin{texcode}
  \cmd{isempty}\marg{content}
\end{texcode}
%
Returns \true\ if \meta{content} is empty.  It is essentially equivalent to
\tex{\cmd{equal}\{\meta{content}\}\{\}} except that the argument of
\cmd{isempty} isn't expanded and therefore isn't affected by fragile
commands\footnote{Internally, it uses \cmd{unexpanded} and \pack{ifmtarg}.}.

It is possible to test whether a~substring appears within another
string\footnote{Uses \cmd{in@} and \cmd{ifin@} internally.}.
%
\begin{texcode}
  \cmd{isin}\marg{substring}\marg{string}
\end{texcode}

Sometimes, you need to check whether a~string ends with a particular
substring.  This can be achieved using\footnote{For compatibility
  reasons, there exist a~command unfortunately called
  \cmd{terminateswith} that performs the same test but it is
  deprecated.}:
%
\begin{texcode}
  \cmd{endswith}\marg{string}\marg{substring}
\end{texcode}

\paragraph{Building more elaborated expressions}
%%%%%%%%%%%%%%%%%%%%%%%%%%%%%%%%%%%%%%%%%%%%%%%%%%%%%%%%%%%%%%%%%

It is then possible to create new tests with:
%
\begin{texcode}
  \cmd{newtest}\marg{command}[$n$]\marg{test expression}
\end{texcode}
%
Surprisingly, a~simple \cmd{newcommand} would not work.  The
\cmd{newtest} macro defines a command \meta{command} taking
$n$~arguments (no optional argument is allowed\footnote{No optional
  argument is allowed because the macro needs to be expanded in the
  first pass and that optional arguments avoid that.} consisting of
the test as specified by \meta{test expression} that can be used in
the argument of \cmd{ifthenelse}.

\section{Examples}

Let's illustrate the most important features of \pack{xifthen} with
the following problem: if we want to test whether a rectangle having
dimensions $l$~and~$L$ meets the two following conditions: $S = l
\times\nobreak L > 100$ and~$P = 2 (l +\nobreak L) < 60$\footnote{Note
  that, because within the arguments of \cmd{cnttest} the \pack{calc}
  is used, you must use real parentheses \tex{(}~and~\tex{)} and not
  \cmd{(}~and~\cmd{)}.}:
%
\begin{texcode}
\cmd{newtest}\{\cmd{condition}\}[2]\{\%\\
\mbox{}\ \ \cmd{cnttest}\{(\#1)*(\#2)\}>\{100\}\%\\
\mbox{}\ \ \cmd{AND}\\
\mbox{}\ \ \cmd{cnttest}\{((\#1)+(\#2))*2\}<\{60\}\%\\
\}
\end{texcode}
%
Then \tex{\cmd{ifthenelse}\{\cmd{condition}\{14\}\{7\}\}\{TRUE\}\{FALSE\}}
returns \ifthenelse{\condition{14}{7}}{TRUE}{FALSE} because $14\times 7 =
98$ and $2\times (14 +\nobreak 7) = 42$, while
\tex{\cmd{ifthenelse}\{\cmd{condition}\{11\}\{11\}\}\{TRUE\}\allowbreak
  \{FALSE\}} returns \ifthenelse{\condition{11}{11}}{TRUE}{FALSE} because
$11 \times 11 = 121$ and $2\times (11 +\nobreak 11) = 44$.

\bigbreak

Now a~list of typical uses of \pack{xifthen}'s capabilities:

\small
\begin{list}{}{\setlength{\leftmargin}{0pt}%
               \setlength{\rightmargin}{-\oddsidemargin}%
               \setlength{\partopsep}{0pt}%
               \setlength{\topsep}{\medskipamount}%
               \setlength{\parsep}{0pt}}%
\item \newcommand*{\TRUE}{%
    \ifthenelse{\equal{\csname f@family\endcsname}{lmtt}}{true}%
      {\makebox[\widthof{false}][l]{\bfseries true}}%
  }
      \newcommand*{\FALSE}{false}
      \newcommand*{\pt}{\ifmmode \,\mathrm{pt}\else pt\fi}

$4 - 1< 4$: \ifthenelse{\cnttest{4 - 1}{<}{4}}{\TRUE}{\FALSE}\quad
$4< 4$:     \ifthenelse{\cnttest{4}{<}{4}}{\TRUE}{\FALSE}\quad
$4 + 1< 4$: \ifthenelse{\cnttest{4 + 1}{<}{4}}{\TRUE}{\FALSE}

$4 - 1\leq 4$: \ifthenelse{\cnttest{4 - 1}{<=}{4}}{\TRUE}{\FALSE}\quad
$4\leq 4$:     \ifthenelse{\cnttest{4}{<=}{4}}{\TRUE}{\FALSE}\quad
$4 + 1\leq 4$: \ifthenelse{\cnttest{4 + 1}{<=}{4}}{\TRUE}{\FALSE}

$4 - 1= 4$: \ifthenelse{\cnttest{4 - 1}{=}{4}}{\TRUE}{\FALSE}\quad
$4= 4$:     \ifthenelse{\cnttest{4}{=}{4}}{\TRUE}{\FALSE}\quad
$4 + 1= 4$: \ifthenelse{\cnttest{4 + 1}{=}{4}}{\TRUE}{\FALSE}

$4 - 1\geq 4$: \ifthenelse{\cnttest{4 - 1}{>=}{4}}{\TRUE}{\FALSE}\quad
$4\geq 4$:     \ifthenelse{\cnttest{4}{>=}{4}}{\TRUE}{\FALSE}\quad
$4 + 1\geq 4$: \ifthenelse{\cnttest{4 + 1}{>=}{4}}{\TRUE}{\FALSE}

$4 - 1> 4$: \ifthenelse{\cnttest{4 - 1}{>}{4}}{\TRUE}{\FALSE}\quad
$4> 4$:     \ifthenelse{\cnttest{4}{>}{4}}{\TRUE}{\FALSE}\quad
$4 + 1> 4$: \ifthenelse{\cnttest{4 + 1}{>}{4}}{\TRUE}{\FALSE}

\medskip

$4\pt - 1\pt< 4\pt$: \ifthenelse{\dimtest{4\pt - 1\pt}{<}{4\pt}}{\TRUE}{\FALSE}\quad
$4\pt< 4\pt$:     \ifthenelse{\dimtest{4\pt}{<}{4\pt}}{\TRUE}{\FALSE}\quad
$4\pt + 1\pt< 4\pt$: \ifthenelse{\dimtest{4\pt + 1\pt}{<}{4\pt}}{\TRUE}{\FALSE}

$4\pt - 1\pt\leq 4\pt$: \ifthenelse{\dimtest{4\pt - 1\pt}{<=}{4\pt}}{\TRUE}{\FALSE}\quad
$4\pt\leq 4\pt$:     \ifthenelse{\dimtest{4\pt}{<=}{4\pt}}{\TRUE}{\FALSE}\quad
$4\pt + 1\pt\leq 4\pt$: \ifthenelse{\dimtest{4\pt + 1\pt}{<=}{4\pt}}{\TRUE}{\FALSE}

$4\pt - 1\pt= 4\pt$: \ifthenelse{\dimtest{4\pt - 1\pt}{=}{4\pt}}{\TRUE}{\FALSE}\quad
$4\pt= 4\pt$:     \ifthenelse{\dimtest{4\pt}{=}{4\pt}}{\TRUE}{\FALSE}\quad
$4\pt + 1\pt= 4\pt$: \ifthenelse{\dimtest{4\pt + 1\pt}{=}{4\pt}}{\TRUE}{\FALSE}

$4\pt - 1\pt\geq 4\pt$: \ifthenelse{\dimtest{4\pt - 1\pt}{>=}{4\pt}}{\TRUE}{\FALSE}\quad
$4\pt\geq 4\pt$:     \ifthenelse{\dimtest{4\pt}{>=}{4\pt}}{\TRUE}{\FALSE}\quad
$4\pt + 1\pt\geq 4\pt$: \ifthenelse{\dimtest{4\pt + 1\pt}{>=}{4\pt}}{\TRUE}{\FALSE}

$4\pt - 1\pt> 4\pt$: \ifthenelse{\dimtest{4\pt - 1\pt}{>}{4\pt}}{\TRUE}{\FALSE}\quad
$4\pt> 4\pt$:     \ifthenelse{\dimtest{4\pt}{>}{4\pt}}{\TRUE}{\FALSE}\quad
$4\pt + 1\pt> 4\pt$: \ifthenelse{\dimtest{4\pt + 1\pt}{>}{4\pt}}{\TRUE}{\FALSE}

\medskip

\tex{\cmd{ifthenelse}\{\cmd{isempty}\{\}\}\{\TRUE\}\{\FALSE\}}
\ifthenelse{\isempty{}}{\TRUE}{\FALSE}

\tex{\cmd{ifthenelse}\{\cmd{isempty}\{ \}\}\{\TRUE\}\{\FALSE\}}
\ifthenelse{\isempty{ }}{\TRUE}{\FALSE}

\tex{\cmd{ifthenelse}\{\cmd{isempty}\{ foo \}\}\{\TRUE\}\{\FALSE\}}
\ifthenelse{\isempty{ foo }}{\TRUE}{\FALSE}

\medskip

\tex{\cmd{ifthenelse}\{\cmd{endswith}\{foo.\}\{.\}\}\{\TRUE\}\{\FALSE\}}
\ifthenelse{\endswith{foo.}{.}}{\TRUE}{\FALSE}

\tex{\cmd{ifthenelse}\{\cmd{endswith}\{foo!\}\{.\}\}\{\TRUE\}\{\FALSE\}}
\ifthenelse{\endswith{foo!}{.}}{\TRUE}{\FALSE}

\medskip

\tex{\cmd{ifthenelse}\{\cmd{isin}\{foo\}\{foobar\}\}\{\TRUE\}\{\FALSE\}}
\ifthenelse{\isin{foo}{foobar}}{\TRUE}{\FALSE}

\tex{\cmd{ifthenelse}\{\cmd{isin}\{Foo\}\{foobar\}\}\{\TRUE\}\{\FALSE\}}
\ifthenelse{\isin{Foo}{foobar}}{\TRUE}{\FALSE}

\medskip

\tex{\cmd{ifthenelse}\{\cmd{cnttest}\{10 * 10 + 1\}>\{100\}\}\{\TRUE\}\{\FALSE\}}
\ifthenelse{\cnttest{10 * 10 + 1}>{100}}{\TRUE}{\FALSE}

\tex{\cmd{ifthenelse}\{\cmd{cnttest}\{10 * 10 + 1\}>\{100 * 100\}\}\{\TRUE\}\{\FALSE\}}
\ifthenelse{\cnttest{10 * 10 + 1}>{100 * 100}}{\TRUE}{\FALSE}

\medskip

\tex{\cmd{ifthenelse}\{\cmd{isequivalentto}\{\cmd{usepackage}\}\{\cmd{RequirePackage}\}\}\{\TRUE\}\{\FALSE\}}
\ifthenelse{\isequivalentto{\usepackage}{\RequirePackage}}{\TRUE}{\FALSE}

\tex{\cmd{ifthenelse}\{\cmd{isequivalentto}\{\cmd{usepackage}\}\{\cmd{textit}\}\}\{\TRUE\}\{\FALSE\}}
\ifthenelse{\isequivalentto{\usepackage}{\textit}}{\TRUE}{\FALSE}

\medskip

\tex{\cmd{ifthenelse}\{\cmd{isnamedefined}\{@foo\}\}\{\TRUE\}\{\FALSE\}}
\ifthenelse{\isnamedefined{@foo}}{\TRUE}{\FALSE}

\tex{\cmd{ifthenelse}\{\cmd{isnamedefined}\{@for\}\}\{\TRUE\}\{\FALSE\}}
\ifthenelse{\isnamedefined{@for}}{\TRUE}{\FALSE}
\end{list}

\end{document}
